\documentclass[12pt,twoside]{article}

\usepackage{amsmath}
\usepackage{color}
\usepackage{enumitem}
\usepackage{graphicx}
\usepackage{booktabs}
\graphicspath{ {images/} }
\usepackage{subfig}
\usepackage{placeins}
\usepackage{float}
\usepackage{mdframed}

\newcommand{\head}[1]{\textnormal{\textbf{#1}}}

\newcommand{\cross}[1][1pt]{\ooalign{%
  \rule[1ex]{1ex}{#1}\cr% Horizontal bar
  \hss\rule{#1}{.7em}\hss\cr}}% Vertical bar

\input{macros}

\setlength{\oddsidemargin}{0pt}
\setlength{\evensidemargin}{0pt}
\setlength{\textwidth}{6.5in}
\setlength{\topmargin}{0in}
\setlength{\textheight}{8.5in}

\newcommand{\theproblemsetnum}{}
\newcommand{\releasedate}{December 1, 2017}
\newcommand{\duedate}{December 1, 2017}
\newcommand{\tabUnit}{3ex}
\newcommand{\tabT}{\hspace*{\tabUnit}}

\usepackage{listings}
\usepackage{color}
\usepackage{amsmath}


\definecolor{dkgreen}{rgb}{0,0.6,0}
\definecolor{gray}{rgb}{0.5,0.5,0.5}
\definecolor{mauve}{rgb}{0.58,0,0.82}

\lstset{frame=tb,
  language=Python,
  aboveskip=3mm,
  belowskip=3mm,
  showstringspaces=false,
  columns=flexible,
  basicstyle={\small\ttfamily},
  numbers=none,
  numberstyle=\tiny\color{gray},
  keywordstyle=\color{blue},
  commentstyle=\color{dkgreen},
  stringstyle=\color{mauve},
  breaklines=true,
  breakatwhitespace=true,
  tabsize=3
}

\title{6.006 Problem Set 4}

\begin{document}

\handout{Crypto project \theproblemsetnum}{\releasedate} %{April 6, 2017}

\setlength{\parindent}{0pt}

\medskip

\hrulefill

\medskip

{\bf Name:} Yihang Yan, Tomas Gudmundsson, Nathaniel Stein

\medskip

{\bf Collaborators:} None

\medskip

\hrulefill

%%%%%%%%%%%%%%%%%%%%%%%%%%%%%%%%%%%%%%%%%%%%%%%%%%%%%
% See below for common and useful latex constructs. %
%%%%%%%%%%%%%%%%%%%%%%%%%%%%%%%%%%%%%%%%%%%%%%%%%%%%%

% Some useful commands:
%$f(x) = \Theta(x)$
%$T(x, y) \leq \log(x) + 2^y + \binom{2n}{n}$
% {\tt code\_function}


% You can create unnumbered lists as follows:
%\begin{itemize}
%    \item First item in a list 
%        \begin{itemize}
%            \item First item in a list 
%                \begin{itemize}
%                    \item First item in a list 
%                    \item Second item in a list 
%                \end{itemize}
%            \item Second item in a list 
%        \end{itemize}
%    \item Second item in a list 
%\end{itemize}

% You can create numbered lists as follows:
%\begin{enumerate}
%    \item First item in a list 
%    \item Second item in a list 
%    \item Third item in a list
%\end{enumerate}

% You can write aligned equations as follows:
%\begin{align} 
%    \begin{split}
%        (x+y)^3 &= (x+y)^2(x+y) \\
%                &= (x^2+2xy+y^2)(x+y) \\
%                &= (x^3+2x^2y+xy^2) + (x^2y+2xy^2+y^3) \\
%                &= x^3+3x^2y+3xy^2+y^3
%    \end{split}                                 
%\end{align}

% You can create grids/matrices as follows:
%\begin{align}
%    A = 
%    \begin{bmatrix}
%        A_{11} & A_{21} \\
%        A_{21} & A_{22}
%    \end{bmatrix}
%\end{align}

\section*{1 - Introduction}
This paper examines the statistical distribution of the price return patterns of cryptocurrencies through the lens of principal component analysis (PCA). The rise of cryptocurrencies is the most intensely debated paradigm shift in financial markets this past decade, with opinions ranging from JPMorgan CEO Jamie Dimon calling them a “fraud” to Bill Gates claiming “Bitcoin is better than currency.” To add to the unfolding drama, the Chicago Board Options Exchange (CBOE) debuted the trading of Bitcoin futures on December 10, 2017. A discussion on the merits and future of these innovations is beyond the scope of this paper.
\bigbreak
Instead, we will assume cryptocurrencies are not going away anytime soon and focus our analysis on other issues pertinent to investors and market-makers in cryptocurrencies. First, we will use PCA to assess whether the price returns of seven major cryptocurrencies move in a parallel manner. Second, we sample different time frames to determine whether there have been any meaningful structural shifts in the price patterns of cryptocurrencies. Finally, we highlight the implications of our results for portfolio construction, risk management, and trading.
\bigbreak
This paper is structured as follows. The following section explains the motivation for applying PCA on financial asset returns and the mathematical theory behind the method before applying it to cryptocurrency price returns. Section 3 will use the results of these PCA findings to address the question of whether cryptocurrencies move in a parallel manner, and Section 4 will contextualize the significance of these findings by assessing whether the price patterns of cryptocurrencies have demonstrated significant inter-temporal changes. Our final section (Section 5) summarizes the conclusions of our analysis. Although PCA is routinely applied to studies of traditional asset classes and incorporated in various econometric models, to the best of this paper’s authors’ knowledge, it has not been applied in as rigorous of a fashion to cryptocurrency price returns.


\section*{2 - Principal Components Analysis}
\subsection*{(a) Motivation}

Before delving into the specifics of this paper’s implementation of PCA, let us motivate the benefits for studying asset returns with PCA. Principal components analysis (PCA) is a multivariate technique that analyzes a data in which observations are described by several inter-correlated quantitative dependent variables. This technique extracts the important information from the data to represent it as a set of new orthogonal variables called principal components, or factors.
\bigbreak
Factor models explaining stock returns and correlations have been very popular in finance. Unlike traditional factor models, the factors created by PCA do not usually have a clear economic interpretation. Rather, the components constructed in PCA are built to have special statistical characteristics whereby each component: 
\begin{itemize}
	\item Aims to account for as much of variation in the data as possible.
	\item Is uncorrelated with every other component, i.e., the components are orthogonal to each other.
\end{itemize}

In short, PCA enables the identification of the underlying statistical factors that cause the comovement in cryptocurrency returns. These findings can be used to quantify the relative importance of each cryptocurrency in explaining the movement of the cryptocurrency market as a whole.
\bigbreak
The first principal component is the component that best explains variation in the underlying data, i.e., the greatest amount of variation, and is of particular importance to this study. In finance, risk is frequently broken down into two categories: \textit{systematic} (i.e., market) risk that can be mitigated through a diversified portfolio and \textit{idiosyncratic} risk that is specific to each individual asset and cannot be diversified away. In applications of PCA on asset returns, the first principal component is generally accepted as a representation of the overall return of the assets, arguably representing the return to investors for taking on the systematic risk of those assets (Shukla and Trzcinka, 1990). In other words, if all the assets shared the same idiosyncratic risks, the first principal component could be conceptualized as the “equal weighted market index” (Shukla and Trzcinka, 1990). Thus, for assets that are fairly correlated with one another, i.e., a large proportion of their comovement is accounted for by the general fortunes (or misfortunes) in their market, one would expect the first component to account for a relatively large proportion of variance and to have similar loadings on the variables.
\bigbreak
Mathematically, PCA depends upon the eigendecomposition of positive semi-definite matrices and upon the singular value decomposition (SVD) of rectangular matrices.

\subsection*{(b) Mathematics of Principal Components}

\subsubsection*{Singular Value Decomposition (SVD)}
\bigbreak
Any real symmetric $m \times m$ matrix A has a spectral decomposition of the form,
$$A = U\triangle U^{T}$$  \hfill(1)
\bigbreak
where $U$ is an orthonormal matrix (matrix of orthogonal unit vectors: $U_{T}U = I$ or $\sum_{k}U_{ki}U_{kj} = \delta_{ij}$) and $\triangle$ is a diagonal matrix. The columns of $U$ are the eigenvectors of matrix $A$ and the diagonal elements of $\triangle$ are the eigenvalues. If $A$ is positive-definite, the eigenvalues will all be positive. Multiplying with $U$, equation 1 can be re-written to,
$$AU = U\triangle U^{T}U = UA$$  \hfill(2)
\bigbreak
This can be written as a normal eigenvalue equation by defining the $i$th column of $U$ as
$u_{i}$ and the eigenvalues as $\lambda_{i} = \triangle_{ii}$:
$$Au_{i} = \lambda_{i}u_{i}$$  \hfill(3)

\bigbreak
Let's look at more general case. An unsymmetrical (n x m) matrix, where $n \geq m$ B has the decomposition,
$$X = U\triangle V^{T}$$  \hfill(4)

where U is a n x m matrix with orthonormal columns ($U^{T}U = I$), while V is a m x m orthonormal matrix ($V_{T}V = I$), and $\triangle$ is a m × m diagonal matrix with positive or zero elements, called the singular values.
\bigbreak
From B we can construct two positive-definite symmetric matrices, $BB^{T}$ and $B^{T}B$, each of which we can decompose
$$BB^{T} = U\triangle V^{T}V\triangle U^{T} = U\triangle^2U^{T} $$  \hfill(5)
$$B^{T}B = V\triangle^2V^{T} $$ \hfill (6)
\bigbreak
We can now show that $BB^{T}$ which is n x n and $B^{T}B$ which is m × m will share m eigenvalues and the remaining n - m eigenvalues of $BB^{T} $ will be zero.
\bigbreak
Using the decomposition above, we can identify the eigenvectors and eigenvalues for $BB_{T}$ as the columns of V and the squared diagonal elements of $\triangle$ , respectively. Denoting one such eigenvector by v and the diagonal element by $\gamma$, we have:

$$B^{T}Bv = \gamma^2v$$ \hfill (7)
$$BB^{T}Bv = \gamma^2Bv$$  \hfill (8)

\bigbreak
This means that we have an eigenvector $u = Bv$ and eigenvalue $\gamma^2$ for $BB^{T}$ as well, since:
\bigbreak
$$(BB^{T})Bv = \gamma^2Bv$$ \hfill (9)

\bigbreak
We have now shown that $B^{T}B$ and $BB^{T}$ share m eigenvalues.
\bigbreak
In order to prove that the remaining n − m eigenvalues of $BB_{T}$ is zero. We need to consider an eigenvector for  $BB^{T}$ , $u_{\perp}$: $BB^{T} u_{\perp} = \beta_{\perp} u_{\perp}$ which is orthogonal to the m eigenvectors $u_{i}$ already determined, i.e. $U^{T} u_{\perp} = 0$. Using the decomposition $BB^{T} = U\triangle^2U^{T}$, we immediately see that the eigenvalues $\beta_{\perp}$ must all be zero,
$$BB^{T} u_{\perp} = U\triangle^2U^{T} u_{\perp} = 0 u_{\perp}$$ \hfill (10)
\bigbreak
\subsubsection*{Principal component analysis (PCA) by SVD}
\bigbreak
We denote the matrix of eigenvectors sorted according to eigenvalue by $\hat{U}$ and we can then PCA transformation of the data as $Y = \hat{U}^{T}X$. The eigenvectors are called the principal components. By selecting the first d rows of $Y$, we can project the data from $n$ down to $d$ dimensions.
\bigbreak
We decompose $X$ using SVD, i.e.
$$X = U\triangle V^{T}$$ \hfill (11)\newline
and find that we can write the covariance matrix as
$$C = \frac{1}{n} XX^{T} = \frac{1}{n} U\triangle^2U^{T}$$ \hfill (12)
\bigbreak
Following from the fact that SVD routine order the singular values in descending order we know that, if $n < m$, the first n columns in $U$ corresponds to the sorted eigenvalues of $C$ and if $m \geq n$, the first m corresponds to the sorted non-zero eigenvalues of $C$. The transformed data can thus be written as:
$$Y = \hat{U}^{T}X = \hat{U}^{T}U\triangle V^T$$ \hfill (13)

where $\hat{U}^{T}U$ is a simple n x m matrix which is one on the diagonal and zero everywhere else. So we can write the transformed data in terms of the SVD decomposition of $X$. 


\subsubsection*{Finding the components}

In PCA, the components are obtained from the singular value decomposition of the dataset $X$. Specifically, with $X = U\triangle V^{T}$ (equation 1), the matrix of factor scores, denoted $F$ is obtained as
$$ F = U\triangle$$ \hfill (14)

The matrix $V$ gives the coefficients of the linear combinations used to compute the factors scores. This matrix can also be interpreted as a projection matrix because multiplying $X$ by $V$ gives the values of the projections of the observations on the principal components. This can be shown as:
$$ F = U\triangle = U\triangle VV^{T}  = XV$$ \hfill (15) 
\bigbreak
The components can be represented geometrically by the rotation of the original axes. Each of these components will be linear combinations of the observed variables we have in our data, and will be orthogonal to each other. That is, each component is independent of each other, and variation in one is unrelated to variation in another. 

\subsubsection*{Contribution of an observation to a component}

The eigenvalue associated to a component is equal to the sum of the squared factor scores for this component. Therefore, the importance of an observation for a component can be obtained by the ratio of the squared factor score of this observation by the eigenvalue associated with that component. This ratio is called the contribution of the observation to the component. Formally, the contribution of observation i to component l is denoted $ ctr_{i,l}$, it is obtained as:
$$ ctr_{i,l} = \frac{f^2_{i,l} }{\sum f^2_{i,l}} = \frac{f^2_{i,l} }{\lambda_{l}} $$  \hfill (16)
where $\lambda_{l}$ is the eigenvalue of the $l$th component. The value of a contribution is between 0 and 1 and, for a given component, the sum of the contributions of all observations is equal to 1. The larger the value of the contribution, the more the observation contributes to the component. 

\section*{3 - Jacobian}

\section*{4 - Implementation}

\subsection*{(a) Data}

Our raw dataset consists of the closing prices for the top seven cryptocurrencies in terms of market capitalization (as listed on CoinMarketCaps' historical tables) for the period between August 7, 2015 to November 7, 2017. The start of the period is the first day for which a closing price on all seven of the selected cryptocurrencies was available because some, like Ripple, did not come into existence until much after the first cryptocurrencies, like Bitcoin. Using this dataset, we first created a time-series matrix containing rolling month-to-month changes in the closing prices for these cryptocurrencies, i.e., rolling monthly returns. For example, the return for Bitcoin (btc) on September 6, 2015 is equal to the percentage change in its closing price from August 7, 2015 (279.58) to September 6, 2015 (239.84): $239.84/279.58 - 1 = -14.21\%$.

\bigbreak
We then standardized the returns for each cryptocurrency by centering around that particular currency's mean return and scaling by its standard deviation to achieve unit variance. The correlation matrix of this standardized dataset serves as the input for PCA. The correlation matrix is typically used instead of the covariance matrix. However, the eigendecomposition of a covariance matrix formed on standardized data yields the same results as an eigendecomposition on a correlation matrix (whether done on standardized or unstandardized data), since the correlation matrix can be understood as the normalized covariance matrix. When using a covariance matrix on unstandardized financial time series data, it should be pointed out that the resulting PCA could lead to the first principal component being dominated by the asset with the largest variance. This is especially impactful when the variances of the individual assets are significantly different from one another.

\bigbreak

\includegraphics[scale=.7]{corr.png}

\bigbreak

One of the basic assumptions behind using PCA is that the variables being examined have some kind of a linear relationship with each other; else, it would be unlikely that they share any meaningful common components. To assess the strength of the variables’ linear relationship and, consequently, suitability for PCA, one can use the correlation coefficients between the pairs of the variables (Hair, 2010). We conclude that the correlation between cryptocurrency pairs is strong enough to at least warrant further exploration.

\subsection*{(b) Computing eigenvectors and corresponding eigenvalues}
\bigbreak
We will use Jacobi method to find eigenvalues and eigenvectors of correlation matrix since this method is a fairly robust way to extract all of the eigenvalues and eigenvectors of a symmetric matrix. The method is based on a series of rotations, called Jacobi or Givens rotations, which are chosen to eliminate off-diagonal elements while preserving the eigenvalues. Details of Jacobi method will be covered in Section xxx.
\bigbreak
We can check if the eigenvector-eigenvalue calculation is correct by using the equation:

$$M_{cov} u_i = \lambda_i u_i$$ where
\newline$M_{cov}$ = covariance matrix, 
\newline$u_i$ = the ith colummn of eigenvector matrix, 
\newline$\lambda_i$ = eigenvalue associated with $u_i$

\subsection*{(c) Sorting Eigenpairs and Explained Variance}

In order to decide which eigenvector(s) can dropped without losing too much information for the construction of lower-dimensional subspace, we need to inspect the corresponding eigenvalues: The eigenvectors with the lowest corresponding eigenvalues bear the least information about the distribution of the data; those are the ones that can be dropped. In order to do so, the common approach is to rank the eigenvalues from highest to lowest in order choose the top $k$ eigenvectors. The first principal component is required to have the largest possible variance. The second component is computed under the constraint of being orthogonal to the first component and to have the largest possible inertia. The other components are computed likewise.
\bigbreak
\begin{tabular}{ccc}
\hline
\head{Principal Component} & \head{Eigenvalue} & \head{Eigenvector}\\
\hline
1  & 2.654 & array([ 0.517,  0.334,  0.358,  0.455,  0.137,  0.487,  0.174])\\
2 & 1.376 &  array([ 0.237,  0.15 , -0.508,  0.309, -0.327,  0.063, -0.676])\\
3 & 0.941 & array([-0.105,  0.305, -0.204, -0.065,  0.88 , -0.037, -0.274])\\
4 & 0.769 & array([-0.286,  0.867,  0.007, -0.226, -0.311, -0.044,  0.13 ])\\
5 & 0.564 & array([-0.015, -0.095,  0.228, -0.611, -0.056,  0.64 , -0.39 ])\\
6 & 0.464 & array([-0.013, -0.064, -0.712, -0.087,  0.027,  0.472,  0.508])\\
7 & 0.239 & array([ 0.764,  0.091, -0.118, -0.511,  0.012, -0.351,  0.103])\\
\hline
\end{tabular}
$$\textbf{Table 1:  Eigenvalue and Eigenvector for each principal component}$$
\bigbreak
After sorting the eigenpairs, the next question is “how many principal components are we going to choose for our new feature subspace?” A useful measure is the proportion of variance, which can be calculated from the eigenvalues. The explained variance tells us how much information (variance) can be attributed to each of the principal components.
\bigbreak
Table 2 displays the results of a principal-component analysis of the cryptocurrencies’ daily returns: The eigenvalues, proportio of variance, and cumulative variance for each component. The single-strongest factor only explains 37.9$\%$ of the variation of crypto-currency returns. Moreover, each subsequent factor is providing only slowly declining additional information content, so that at least 5 factors are needed in order to account for 90$\%$ of the variation from these 7 crypto-currencies. 
\bigbreak
\begin{tabular}{cccc}
\hline
\head{Principal Component} & \head{Eigenvalue} & \head{Proportion of Variance} & \head{Cumulative Variance}\\
\hline
1 & 2.654 &  37.915\% & 37.915\%\\
2 & 1.376 & 19.635\% & 57.55\%\\
3 & 0.941 &  13.422\% & 70.972\%\\
4 & 0.769 &  10.967\% & 81.939\%\\
5 & 0.564 &  8.040\%  & 89.979\%\\
6 & 0.464 & 6.613\% & 96.592\%\\
7 & 0.239 &  3.408\% & 100\%\\
\hline
\end{tabular}
$$\textbf{Table 2:  Explained variance and cumulative varaince for each principal component}$$
\bigbreak
\textbf{??????
ADD (Comparison with XLE)}
\bigbreak

\subsection*{(d) Component Loadings}
\bigbreak
In multivariate space, the correlation between the principal component and the original variables (cryptocurrency price returns) is called the loading (or weight) of that component on the original variable. Based on loadings, we can tell how much of the variation in a variable is explained by the component.
\bigbreak
Loadings = Orthonormal Eigenvectors $* \sqrt{Absolute Eigen values}$
\bigbreak
Principal component loading diagram is shown below:
\bigbreak
\includegraphics[scale=.6]{pca_loadings.png}
\bigbreak
\begin{tabular}{cccccccc}
\hline
\head{Principal \newline Component} & \head{Ripple} & \head{Bitcoin} & \head{Litecoin} &\head{Monero} &\head{Nem} &\head{Dash} &\head{Ethereum}\\
\hline
1 & 0.843&0.544&0.583&0.741&0.224&0.793&0.284\\
2 & 0.277&0.176&-0.595&0.363&-0.383&0.074	&-0.793\\
3 & -0.102	&0.295&-0.198&-0.063&0.853&-0.036&-0.266\\
4 & -0.250&0.760&0.006&-0.198&-0.272&-0.039&0.114\\
5 & -0.011	&-0.071&0.171&-0.459&-0.042&0.480&-0.293\\
6 & -0.009&-0.043&-0.484&-0.059&0.018&0.321&0.345\\
7 & 0.373&0.044&-0.058&-0.249&0.006&-0.171&0.050\\
\hline
\end{tabular}
$$\textbf{Table 3: Component Loadings}$$

\bigbreak
The first principal component is strongly correlated (loading score $\geq$ 0.7) with three out of seven currencies. The first principal component increases with increasing Ripple, Monero, and Dash scores. This suggests that these three currencies are likely to vary together (positively correlated). If one increases, then the remaining ones tend to as well. Furthermore, we see that the correlation of first principal component with those six currencies are quite similar. \newline
We also notice that the second principal component (negatively) correlates most strongly with Ethereum. In fact, we could state that based on the correlations of -0.793, this principal component is primarily a measure of Ethereum.
\bigbreak
We construct the bi-plot of relative weights of each cryptocurrency in the first two PC components (PC-1 and PC-1) arising from the previous analysis:
\bigbreak
\includegraphics[scale=.6]{biplot1.png}
\bigbreak
This diagram also demonstrate the distinct movement between Ethereum and the rest of cryptocurrencies. All seven variables have positive values in the PC1 axis, while Litecoin, Nem, and Ethereum are negative in PC2's and Ripple, Bitcoin, Monero and Dash is positive. Since all the variables are positive in PC1, those which constrain the system the most are Ripple and (then) Dash and Monero (in PC1 axis).The PC2, which has much smaller variance, contrasts Ethereum from everything else.
\bigbreak
The cosine of the angles between vectors is equal to the correlation between those variables. Hence vectors pointing in the same direction are perfectly correlated, and those at right angles are uncorrelated. As we can see, Ripple and Bitcoin are high correlation, while Ethereum has extreme low correlation with Bitcoin, Monero, and Ripple. This conclusion is also supported by the correlation matrix.

\bigbreak

The purpose of this section was to elucidate how PCA can be conducted on a financial time series like the universe of cryptocurrencies that are the focus of this paper. We have shown what the eigenvectors, components and loadings look like. Yet, we have not provided any point of reference against which to contextualize these results. This will be done in the next section, where we aim to answer the question of whether cryptocurrencies move in a parallel manner. To answer such a question, one needs to define what return patterns would justify being considered "parallel.". To do this, we will employ statistical tests done in other PCA studies on financial time series data and provide a point of reference to help compare and contrast the results of PCA on cryptocurrencies against another asset class.

In order to further confirm the weak correlation for each pair of currencies (cryptocurrency time-series), we make use of two distinct tools, namely, one-factor linear regression (hence its $R^2$ metric) and Kendall’s rank correlation metric of $\tau$. (NOT SURE IF WE NEED THIS.. DEPEND ON ANY SPACE LEFT)

\section*{5 - Do Cryptocurrencies Move in a Parallel Manner?}

On November 23, 2017, Fortune magazine ran the story titled "Ethereum Price Hits New High as Billionaire Predicts 25\% Surge In the Next Month." Several articles like this appear daily for seemingly each cryptocurrency, and each touts the purported benefits of its subject coin's features over the rest. This all alludes to a question common in the investment world for all groupings of assets. If Ethereum does well in a given month or year, is it because \textit{Ethereum} is special in relation to its coin colleagues or is it because cryptocurrencies \textit{as a whole} did well? In more statistical words, how much of the variation in individual cryptocurrencies could be described by variation in the universe of cryptocurrencies as a whole? And relatedly, is it useful to aggregate cryptocurrencies into an "asset class" or grouping? Before answering this ourselves, we review prior related studies on the matter.
\subsection*{(a) Existing Literature}
Broadie (2012) conducts PCA on the price returns of assets that are generally accepted to rise and fall in a parallel fashion as a result of changes in prevailing and expected future interest rates: on-the-run Treasury securities of varying maturities. He finds that not only does the first principal component account for an overwhelming majority of the variation (95\%), but also that the loadings of the first component are near-equal with respect to each individual Treasury security. Of course, these findings are not surprising given that the most significant risk to the price of Treasury securities, i.e., cause for variation in their price returns, is a change in the interest rate curve the effect it has on the discounting of cash flows. Unlike with Treasuries, the causes for variation in the return on equities are arguably more nuanced. Feeney and Hester (1967), in one of the first studies of its kind, conduct PCA on the rates of return of the 30 Dow Jones industrial (DJI) stocks over a 50 quarter period and find that the first component accounts for 41\% of the variance (the second component accounts for 9\%) and loads positively on each of the 30 stocks. They conclude that the returns of individual stocks are "dominated by the tone of the market." Borrowing their terminology, we can rephrase the question addressed by this section: is there a significant "tone of the market" for cryptocurrencies?
\subsection*{(b) Proportion test}
To answer these questions, we employ a rough test similar to the one Feeney and Hester (1967) use to determine whether it is useful to aggregate stocks into industries. If such an aggregation were useful, they postulate that we would expect stocks in the same industry to enter different components of the market with similar weights, i.e., for the stocks to have similar correlations to different components of market returns. Similarly, we aim to discover whether the cryptocurrencies enter the components of the cryptocurrency market returns with similar weights. We conduct a proportions test on a null hypothesis that the correlations of two cryptocurrencies with a component (i.e., their loadings with respect that component) will be of the same sign with a probability of 0.5 (with a one-sided right-tail rejection region). Since we will be limiting our analysis to the number of components that it takes to explain more than 70\% of the variation in cryptocurrency returns (3), the test is performed on the pooled sum of the 63 pairwise comparisons at a 5\% significance level (seven cryptocurrencies in pairs of two across three components: $\choose{7}{2} \times 3$. Like Feeney and Hester (1967), we will assume the signs are binomially distributed in order to use this test.
\bigbreak
In other words, the more frequently a pair of cryptocurrencies have correlation coefficients of the same sign with the various components, the more justified we should feel in conceptually aggregating cryptocurrencies into one investment class (hence the right-tailed rejection region). Rejecting this hypothesis would imply that an investor could reduce the risk in their portfolio by spreading out their exposure to cryptocurrencies among multiple currencies, just as one might spread out their exposure to equities among many stocks. The test yields a $z$-statistic of 2.394, which is statistically significant at the 1\% level. Consequently, we reject the null hypothesis and conclude that investing in more than one currency would provide no advantage from the perspective of building a diversified portfolio to reduce overall variance (unless, perhaps, you were to “short” one or more of the other currencies). The results of this test lead us to believe with more conviction that we are justified in treating cryptocurrencies like an investment class.
\bigbreak
Although this was a rough test, we elucidate the results by providing two points of reference. We conduct the same proportions test that we did on the cryptocurrencies to two other sets of seven assets. The first set consists of the top seven holdings of the Energy Select Sector SPDR® Fund (Bloomberg Ticker: XLE), which "seeks to provide an effective representation of the energy sector of the S\&P 500 Index." The energy sector is often described as one in which the stocks are highly correlated with one another, with banks such as Goldman Sachs and JP Morgan often making sector-wide recommendation calls (i.e., buy or sell the "energy stocks") depending on the tides of oil prices. The second set of seven assets was selected with the deliberate intent of including not only stocks of many industries but exchange-traded funds representing various industries as well. Tables \ref{table:1} and \ref{table:2} provide more details on the contents of both sets of assets as well as their loading results. We performed the test on each of these sets with a number of components equal to the number required to account for more than 70\% of the variance and comparer the results in table \ref{table:3}, where $\sigma^2$ refers to the variance accounted for by the number of components used.
\bigbreak

\begin{table}[h!]
	\centering
\begin{tabular}{lrrrrrrr}
	\toprule
Component &   AAPL &    GLD &    HYG &      X &     GE &      M &    MOS \\
\midrule
1         &  0.825 &  0.809 & -0.459 & -0.450 & -1.074 & -0.135 &  0.227 \\
2         &  0.057 & -0.421 & -0.185 &  0.572 & -0.486 &  0.091 & -0.193 \\
3         &  0.017 & -0.069 &  0.017 & -0.193 & -0.036 & -0.107 & -0.393 \\
4         & -0.207 &  0.208 & -0.024 &  0.094 & -0.050 & -0.000 & -0.088 \\
5         &  0.086 &  0.055 &  0.122 &  0.072 &  0.017 &  0.005 & -0.039 \\
6         & -0.046 & -0.026 &  0.081 & -0.024 & -0.066 & -0.049 &  0.037 \\
7         & -0.006 & -0.002 &  0.013 & -0.017 & -0.013 &  0.052 & -0.004 \\
	\bottomrule
\end{tabular}
\caption{PCA loadings for diverse mix of stocks and exchange-traded funds: Apple Inc. (AAPL), SPDR Gold Shares ETF (GLD),  iShares iBoxx \$ High Yid Corp Bond ETF (HYG), United States Steel Corporation (X), General Electric (GE), Macy's Inc (M), Mosaic Co (MOS) }
\label{table:1}
\end{table}

\begin{table}[h!]
	\centering
	\begin{tabular}{lrrrrrrr}
		\toprule
		Component &   COP &    CVX &    EOG &    OXY &    XOM &    SLB &    PSX \\
		\midrule
1         & -0.996 & -0.959 & -0.886 & -0.861 & -0.541 & -0.812 & -0.860 \\
2         & -0.672 & -0.035 & -0.092 &  0.383 & -0.269 &  0.226 &  0.483 \\
3         &  0.146 &  0.353 & -0.351 & -0.163 & -0.131 & -0.366 &  0.390 \\
4         &  0.226 & -0.043 &  0.108 &  0.126 & -0.577 &  0.000 & -0.089 \\
5         & -0.013 & -0.142 &  0.208 &  0.170 &  0.069 & -0.355 &  0.081 \\
6         &  0.143 & -0.141 & -0.282 &  0.255 &  0.075 &  0.015 & -0.034 \\
7         & -0.100 &  0.234 & -0.017 &  0.144 &  0.003 & -0.076 & -0.201 \\
		\bottomrule
	\end{tabular}
	\caption{PCA loadings for top seven holdings of the Energy Select Sector SPDR® Fund: ConocoPhillips (COP), Chevron Corporation (CVX), EOG Resources Inc (EOG), Occidental Petroleum Corporation (OXY), Exxon Mobil Corporation (XOM), Schlumberger Limited. (SLB), Phillips 66 (PSX).}
	\label{table:2}
\end{table}


\begin{table}[h!]
	\centering
\begin{tabular}{lrrr}
	\toprule
	{} &  Crypto &  Energy &  Basket \\
	\midrule
	Components &   3 &   2 &   1 \\
	$z$        &   2.394 &   2.777 &  -0.655 \\
	$p$-value  &   0.008 &   0.003 &   0.744 \\
	$\sigma^2$ &   0.710 &   0.795 &   0.714 \\
	\bottomrule
\end{tabular}
\caption{Proportions test on three sets of assets}
\label{table:3}
\end{table}

\subsection*{(c) Inter-temporal consistency of covariance matrix}

Of course, this analysis will be relevant in future years only if the covariance matrix for cryptocurrency returns remains relatively unchanged over time. It may be the case that cryptocurrencies that are highly correlated with each other in recent history may become less correlated in the future, raising the scepter of a diversification benefit to purchasing multiple cryptocurrencies. Although we cannot look into the future, we can investigate the inter-temporal consistency of the covariance matrix by splitting the recent history we have available into two samples.



The proportion of variance is significantly higher in the second time period. Building a regression model to forecast the loadings in the second time period based on the loadings from the first time period is unsuccessful.

Unfortunately, then, any portfolio considerations formed should be utilized with caution given the changing nature of the covariance matrix.

[to-do next: try month-to-month samples]

\end{document}







