\documentclass[12pt,twoside]{article}

\usepackage{amsmath}
\usepackage{color}
\usepackage{enumitem}
\usepackage{graphicx}
\graphicspath{ {images/} }

\usepackage{subfig}
\usepackage{placeins}
\usepackage{float}


\usepackage{mdframed}


\newcommand{\cross}[1][1pt]{\ooalign{%
  \rule[1ex]{1ex}{#1}\cr% Horizontal bar
  \hss\rule{#1}{.7em}\hss\cr}}% Vertical bar

\input{macros}


\setlength{\oddsidemargin}{0pt}
\setlength{\evensidemargin}{0pt}
\setlength{\textwidth}{6.5in}
\setlength{\topmargin}{0in}
\setlength{\textheight}{8.5in}

\newcommand{\theproblemsetnum}{}
\newcommand{\releasedate}{December 1, 2017}
\newcommand{\duedate}{December 1, 2017}
\newcommand{\tabUnit}{3ex}
\newcommand{\tabT}{\hspace*{\tabUnit}}

\usepackage{listings}
\usepackage{color}
\usepackage{amsmath}


\definecolor{dkgreen}{rgb}{0,0.6,0}
\definecolor{gray}{rgb}{0.5,0.5,0.5}
\definecolor{mauve}{rgb}{0.58,0,0.82}

\lstset{frame=tb,
  language=Python,
  aboveskip=3mm,
  belowskip=3mm,
  showstringspaces=false,
  columns=flexible,
  basicstyle={\small\ttfamily},
  numbers=none,
  numberstyle=\tiny\color{gray},
  keywordstyle=\color{blue},
  commentstyle=\color{dkgreen},
  stringstyle=\color{mauve},
  breaklines=true,
  breakatwhitespace=true,
  tabsize=3
}

\newcommand{\dd}[1]{\mathrm{d}#1}

\title{6.006 Problem Set 4}

\begin{document}

\handout{Crypto project \theproblemsetnum}{\releasedate} %{April 6, 2017}

\setlength{\parindent}{0pt}

\medskip

\hrulefill

\medskip

{\bf Name:} Yihang Yan, Tomas Gudmundsson, Nathaniel Stein

\medskip

\hrulefill

%%%%%%%%%%%%%%%%%%%%%%%%%%%%%%%%%%%%%%%%%%%%%%%%%%%%%
% See below for common and useful latex constructs. %
%%%%%%%%%%%%%%%%%%%%%%%%%%%%%%%%%%%%%%%%%%%%%%%%%%%%%

% Some useful commands:
%$f(x) = \Theta(x)$
%$T(x, y) \leq \log(x) + 2^y + \binom{2n}{n}$
% {\tt code\_function}


% You can create unnumbered lists as follows:
%\begin{itemize}
%    \item First item in a list 
%        \begin{itemize}
%            \item First item in a list 
%                \begin{itemize}
%                    \item First item in a list 
%                    \item Second item in a list 
%                \end{itemize}
%            \item Second item in a list 
%        \end{itemize}
%    \item Second item in a list 
%\end{itemize}

% You can create numbered lists as follows:
%\begin{enumerate}
%    \item First item in a list 
%    \item Second item in a list 
%    \item Third item in a list
%\end{enumerate}

% You can write aligned equations as follows:
%\begin{align} 
%    \begin{split}
%        (x+y)^3 &= (x+y)^2(x+y) \\
%                &= (x^2+2xy+y^2)(x+y) \\
%                &= (x^3+2x^2y+xy^2) + (x^2y+2xy^2+y^3) \\
%                &= x^3+3x^2y+3xy^2+y^3
%    \end{split}                                 
%\end{align}

% You can create grids/matrices as follows:
%\begin{align}
%    A = 
%    \begin{bmatrix}
%        A_{11} & A_{21} \\
%        A_{21} & A_{22}
%    \end{bmatrix}
%\end{align}
% http://www.aip.de/groups/soe/local/numres/bookcpdf/c11-1.pdf
\section*{1}
\subsection*{Jacobi method}
In order to find eigenvalues and eigenvectors of the covariance matrix, A, we use the Jacobi method. The method finds the largest element below the diagonal in the matrix at location $(i,j)$ and simulates a rotation with A as follows:
\begin{equation}
     A' = P_{i,j,\theta}^T \cdot A \cdot P_{i,j,\theta}  
\end{equation}

where $P_{i,j,\theta} = P(i,j,\theta)$ is a Givens rotation matrix with the form:

\begin{equation}
P(i,j,\theta) = 
\begin{bmatrix}
     1 & \cdots & 0 & \cdots & 0 & \cdots & 0 \\
     \vdots & \ddots & \vdots &   & \vdots &  & \vdots \\
          0 & \cdots & c & \cdots & s & \cdots & 0 \\
         \vdots &  & \vdots & \ddots  & \vdots &  & \vdots \\
     0 & \cdots & -s & \cdots & c & \cdots & 0 \\
              \vdots &  & \vdots &  & \vdots & \ddots  & \vdots \\
                   0 & \cdots & 0 & \cdots & 0 & \cdots & 1 \\
\end{bmatrix}
\end{equation}

This matrix has ones on the diagonal except where $c=cos(\theta)$ at locations (i,i) and (j,j). All other elements are 0 except $s=sin(\theta)$ at locations (i,j) and (j,i).\\

When the Givens rotation matrix, $P_{i,j,\theta}$,  is multiplied with another matrix, $A$, as $P\cdot A$ it simulates a clockwise rotation in the plane by an angle $\theta$ in order to nullify the element at location $(i,j)$ and only affects rows and columns $i$ and $j$ in the process.\\
 

In order to figure out the values of $c,s$ and $\theta$ we will simulate a matrix multiplication with $2x2$ matrices that contain the relevant information. We will represent the matrices as follows:
\begin{equation}
P_{i,j,\theta} = 
\begin{bmatrix}
     c_{ii} & s_{ij} \\
    -s_{ji} & c_{jj} \\
\end{bmatrix}
,A = 
\begin{bmatrix}
     A_{ii} & A_{ij} \\
    A_{ji} & A_{jj} \\
\end{bmatrix}
,A' = 
\begin{bmatrix}
     A_{ii}' & A_{ij}' \\
    A_{ji}' & A_{jj}' \\
\end{bmatrix}
\end{equation}

where the notation $A_{ij}$ denotes element at location $(i,j)$ in matrix A.


Now we will find $c, s$ and $\theta$ so that the matrix multiplication nullifies the largest element. In order for P to satisfy eq. 1 we expand the matrix multiplication as follows with $c_{ii}=c_{jj}=c$ and $s_{ji}=s_{ij}=s$:\\
\begin{equation}
\begin{split}
\begin{bmatrix}
     A_{ii}' & A_{ij}' \\
    A_{ji}' & A_{jj}' \\
\end{bmatrix}
 &=
\begin{bmatrix}
     c & -s \\
    s & c \\
\end{bmatrix}
\cdot
\begin{bmatrix}
     A_{ii} & A_{ij} \\
    A_{ji} & A_{jj} \\
\end{bmatrix}
\cdot
\begin{bmatrix}
     c & s \\
     -s & c \\
\end{bmatrix}\\
&=\begin{bmatrix}
     c\cdot A_{ii} - s\cdot A_{ji} & c\cdot A_{ij} - s\cdot A_{jj} \\
     s\cdot A_{ii} + c\cdot A_{ji} & s\cdot A_{ij} + c\cdot A_{jj} \\
\end{bmatrix}
\cdot
\begin{bmatrix}
     c & s \\
     -s & c \\
\end{bmatrix}\\
&=\begin{bmatrix}
c^2\cdot A_{ii} - cs\cdot A_{ji} - cs\cdot A_{ij} + s^2\cdot A_{jj} & cs\cdot A_{ii} - s^2\cdot A_{ji} + c^2\cdot A_{ij} - cs\cdot A_{jj}\\
cs\cdot A_{ii} + c^2\cdot A_{ji} - s^2\cdot A_{ij} - cs\cdot A_{jj} & s^2\cdot A_{ii} + cs\cdot A_{ji} + cs\cdot A_{ij} + c^2\cdot A_{jj}\\
\end{bmatrix}\\
&=\begin{bmatrix}
c^2\cdot A_{ii} - 2cs\cdot A_{ij}  + s^2\cdot A_{jj} & (c^2-s^2)\cdot A_{ij}  + cs\cdot (A_{ii} - A_{jj})\\
(c^2-s^2)\cdot A_{ij}  - cs\cdot (A_{ii} - A_{jj}) & c^2\cdot A_{jj} + 2cs\cdot A_{ij} + s^2\cdot A_{ii}\\
\end{bmatrix}
\end{split}
\end{equation}

where the last elements are simplified because $A_{ij}=A_{ji}$.\\



In order to make the non diagonal element in this matrix as 0 we will examine the non diagonal equation as follows:
\begin{equation}
A'_{ij} = (c^2 - s^2) \cdot A_{ij} + cs(A_{ii}-A_{jj}) = 0
\end{equation}


Hence it follows that:
\begin{equation}
\frac{c^2-s^2}{cs} = \frac{A_{jj}-A_{ii}}{A_{ij}}
\end{equation}

and we can define the rotation angle as follows:
\begin{equation}
\theta = cot(2\phi) = \frac{c^2-s^2}{2cs} = \frac{A_{jj}-A_{ii}}{2A_{ij}}
\end{equation}
and by letting $t = s/c$ we can rewrite the equation above as:\\
\begin{equation}
2cs\theta = c^2 - s^2 <=>
t^2 + 2t\theta - 1 = 0
\end{equation}
which has the solutions:\\
\begin{equation}
t = 
\bigg\{
  \begin{tabular}{cc}
$ -\theta + \sqrt{\theta^2+1}$ \\
$ -(\theta + \sqrt{\theta^2+1})$ \\
  \end{tabular}
\end{equation}
These solutions can be written more succinctly as
\begin{equation}
t =  -\theta + \sqrt{\theta^2+1}  = \frac{ \left(-\theta + \sqrt{\theta^2+1}\right)\left(-\theta + \sqrt{\theta^2+1}\right) }{\theta + \sqrt{\theta^2+1}}
= \frac{ -\theta^2 + \theta^2 + 1 }{\theta + \sqrt{\theta^2+1}} = \frac{1}{\theta + \sqrt{\theta^2+1}}
\end{equation}
\begin{equation}
t =  -\theta - \sqrt{\theta^2+1}  = \frac{ \left(-\theta - \sqrt{\theta^2+1}\right)\left(\theta - \sqrt{\theta^2+1}\right) }{\theta - \sqrt{\theta^2+1}}
= \frac{ -\theta^2 + \theta^2 + 1 }{\theta - \sqrt{\theta^2+1}} = \frac{-1}{-\theta + \sqrt{\theta^2+1}}
\end{equation}

We want to rotate the matrix by the angle which corresponds to the smaller root of this equation and generally we can write the smaller root as:

\begin{equation}
t =  \frac{sign(\theta)}{|\theta|+ \sqrt{1+\theta^2}}
\end{equation}
and since $t=s/c$ we now have:\\
\begin{equation}
c =  \frac{1}{\sqrt{t^2+1}}, \hspace{5mm} s = t\cdot c
\end{equation}


Now we know how to set these variables for the rotations to work and we need to look at three scenarios to update the matrix when a rotation is performed:

\begin{enumerate}[label=\roman*)]
  \item Set value at location (i,j) as 0
  \item Change diagonal values at locations (i,i) and (j,j)
  \item Change values on rows and columns i and j except (i,i) and (j,j)
\end{enumerate}

For scenario i) we simply set the value of $A_{ij}'$ as 0. However, for scenario ii) we will look at the top left and bottom right elements in eq. 4 to gather equations to set the diagonal elements $A_{ii}'$ and $A_{jj}'$ . We have:
\begin{equation}
A_{ii}' = c^2 \cdot A_{ii} - 2cs\cdot A_{ij} + s^2\cdot A_{jj}
\end{equation}

From eq. 5 (because $A_{ij}'=0$) we can isolate $A_{jj}$ as\\
\begin{equation}
A_{jj} = A_{ii} - A_{ij}\frac{s^2-c^2}{cs}
\end{equation}
and since $c^2+s^2=1$ we simplify eq 14. as:

\begin{equation}
\begin{split}
A_{ii}' &= c^2 \cdot A_{ii} - 2cs\cdot A_{ij} + s^2\cdot A_{jj}\\
& = c^2 \cdot A_{ii} - 2cs\cdot A_{ij} + s^2 \left(A_{ii} - A_{ij}\frac{s^2-c^2}{cs}     \right)\\
& = (c^2 + s^2) \cdot A_{ii} - s\left(2c +  \frac{s^2-c^2}{c}      \right)A_{ij}\\
&= (c^2 + s^2) \cdot A_{ii} - \frac{s}{c}\left(2c^2 +  s^2-c^2      \right)A_{ij}\\
& = A_{ii} - \frac{s}{c}\left(c^2 +  s^2      \right)A_{ij}\\
& = A_{ii} - t\cdot A_{ij} 
\end{split}
\end{equation}
Similarly we have:\\
\begin{equation}
A_{jj}' = A_{jj} + t\cdot A_{ij}
\end{equation}



For scenario iii) we can look at top of eq 4. and note that if we consider an element $A_{rj}$ when we perform rotation around $A_{ij}$ that only the last two matrices will change the result since the first matrix changes rows i and j and does not have effect on row $r$. The last matrix changes columns i and j and therefore changes the resulting matrix. Multiplying through these matrices gives us the equations:
\begin{equation}
\bigg\{
  \begin{tabular}{cc}
$A_{ri}' = cA_{ri} - sA_{rj}$\\
$A_{rj}' = cA_{ri} + sA_{rj}$\\
  \end{tabular}
\end{equation}\\

Lets look at $A_{ri}'$ which can be represented as:
\begin{equation}
\begin{split}
A_{ri}' &= cA_{ri} - sA_{rj}\\
&= \left(1 - \frac{(1-c)(1+c)}{1+c} \right) A_{ri} - sA_{rj} \\
&= \left(1 - \frac{1-c^2}{1+c} \right) A_{ri} - sA_{rj} \\
&= \left(1 - \frac{s^2}{1+c} \right) A_{ri} - sA_{rj} \\
&= A_{ri}  - s \left(A_{rj} + \frac{s}{1+c} A_{ri} \right) \\
&= A_{ri}  - s \left(A_{rj} + \tau A_{ri} \right)  
\end{split}
\end{equation}

where 
\begin{equation}
\tau = \frac{s}{1+c}
\end{equation}

Similarly we have
\begin{equation}
A_{rj}' = A_{rj}  + s \left(A_{ri} - \tau A_{rj} \right) 
\end{equation}

\vspace{5mm}
To summarise we set values of elements in rows r and l and columns r and l as follows:
\begin{enumerate}[label=\roman*)]
  \item $A_{ij}=0$
\item $\bigg\{
  \begin{tabular}{cc}
$A_{ii}' = A_{ii} - t\cdot A_{ij}$  \\
$A_{jj}' = A_{jj} + t\cdot A_{ij}$ \\
  \end{tabular}$
\item $\bigg\{
  \begin{tabular}{cc}
$A_{ri}' = A_{ri}  - s \left(A_{rj} + \tau A_{ri} \right)$   \\
$A_{rj}' = A_{rj}  + s \left(A_{ri} - \tau A_{rj} \right)$   \\
  \end{tabular}$
\end{enumerate}

where
\[ s=t\cdot c, \hspace{5mm} t = \frac{sign(\theta)}{|\theta|+ \sqrt{1+\theta^2}}, \hspace{5mm} \tau = \frac{s}{1+c}, \hspace{5mm} \theta = \frac{A_{ii}-A_{jj}}{2A_{ij}}  \]






\end{document}







